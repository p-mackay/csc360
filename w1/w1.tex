\documentclass{article} % Use the custom resume.cls style
\usepackage{enumitem}
\usepackage{multicol}
\usepackage{listings}
\renewcommand{\familydefault}{\sfdefault}

\title{Written Assignment 1 (W1) CSc360 OS}
\author{Paul MacKay V00967869}

\begin{document}
\maketitle

\begin{enumerate}
    \item
        \begin{enumerate}[label=(\alph*)]
            \item
                \begin{itemize}
                    \item While in user mode (mode bit = 1), the user does not have direct access to the 
                        operating system. Meaning - a user program must ask the operating system
                        first before the application is ran. If the operating system validates that
                        it is a legitimate program then the operating system executes it by 
                        switching to kernel mode (mode bit = 0).
                    \item When in kernel mode the program has the privilege to execute. 
                \end{itemize}
                \hfill
            \item
                \begin{itemize}
                    \item For example if an operating system did not have two modes
                        (user and kernel) then it would be possible for the user
                        to wipe out the entire operating system by overwriting 
                        it, or delete it.
                \end{itemize}
                \hfill
            \item
                \begin{itemize}
                    \item Mode switching is when an operating system switches between 
                        user or kernel modes. Which protects the operating system from
                        getting corrupted or deleted.
                    \item   
                        Context switching is when the CPU switches from one process to another.
                        This is achieved by saving the state of the current process, and 
                        continues to execute a different process.
                \end{itemize}
                \hfill
            \item
                \begin{itemize}
                    \item             
                        Pros: Because the microkernel is smaller: (1) it makes it easier to port 
                        from one system to another. (2) Its more secure. 
                    \item   
                        Because of system overhead the performance of microkernels can suffer.
                        For example Windows NT(used a microkernel) performance 
                        was worse than Windows 95(didn't use microkernel).
                \end{itemize}
                \hfill
        \end{enumerate}
        \newpage
    \item
        \begin{enumerate}
            \item
                \begin{enumerate}[label={}]
                        Possible output:
                        \begin{multicols}{2}
                        \item 0
                        \item 1 
                        \item 2
                            \columnbreak
                        \item 0
                        \item 2
                        \item 1
                        \end{multicols}
                \end{enumerate}
                \hfill
            \item
                \begin{lstlisting}[language=C]
    #define OUTPUT printf("%d\n", i)
    main() {
        int i=0;
        OUTPUT;
        if (fork()) {
            i+=2;
            OUTPUT;
        } else {
            i+=1;
            OUTPUT;
            wait(NULL); /*system call*/
            return(0);
        }
    }
                \end{lstlisting}
        \end{enumerate}
        \hfill
    \item
        \begin{enumerate}[label=(\alph*)]
            \item
                        running-to-waiting
                \begin{itemize}
                    \item 
                        Given that only one process can be running at a time.
                        The processor must switch between running-waiting-ready...
                        frequently.
                    \item 
                        An example of this would be if a process is running
                        then the scheduler issues an I/O request. Or The 
                        process creates a child process then waits for the 
                        child process to execute.
                \hfill
                \end{itemize}
            \item
                        waiting-to-running
                \begin{itemize}
                    \item 
                        Not feasible. Since the ready queue uses a linked list
                        to store all ready processes. The process would not 
                        be added to the linked list and would be lost, and would
                        never run.
                \end{itemize}
                \hfill
            \item
                        waiting-to-ready
                \begin{itemize}
                    \item 
                        After the I/O request is finished. Or the Child process
                        is done executing. We return back to the ready state.
                \end{itemize}
                \hfill
            \item
                        ready-to-waiting
                \begin{itemize}
                    \item 
                        Not feasible. Since the process is in the ready state. 
                        It is in the ready queue waiting to be executed. This means
                        that it doesn't have the ability to issue a child process,
                        or an I/O request.
                \end{itemize}
                \newpage
            \item
                        ready-to-running
                \begin{itemize}
                    \item 
                        The CPU scheduler chooses the process from the 
                        ready queue to be executed.
                \end{itemize}
                \hfill
            \item
                        running-to-ready
                \begin{itemize}
                    \item 
                        Once a process is finished executing an interrupt is issued
                        to communicates to the CPU that the process is finished 
                        executing.
                \end{itemize}
                \hfill
        \end{enumerate}

\end{enumerate}


\end{document}

