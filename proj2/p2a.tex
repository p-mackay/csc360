\documentclass[10pt]{article}
\usepackage[letterpaper, margin=1in]{geometry}
\renewcommand{\familydefault}{\sfdefault}
\begin{document}

\title{Assignment: p2a}
\author{Paul Mackay V00967869}
\date{\today}

\maketitle

\begin{enumerate}
    \item 
        \begin{enumerate}
            \item Main thread (int main()) 
                \begin{enumerate}
                    \item Read input file: read a text file that contains
                        priority, direction, loading time, and crossing time.
                    \item Create train threads 
                        in a loop create train threads using pthread\_create, then
                        at the end use pthread\_join.
                    \item Initialize data structures: 
                        (1) Initialize train structs, each train has the following attributes: number,
                        direction, priority, loadingTime, and crossingTime.
                        (2) Initialize train priority queue.
                        (3) Initialize Mutexes.
                        (4) Initialize Condition Variables.
                        (5) Initialize status flag for if all trains have crossed.
                    \item Manage the simulation time.
                        Track the current state of each train: ready, ON, OFF. 
                    \item Coordinate the termination of the program once all trains have crossed.
                        if all trains have passed then exit program.
                \end{enumerate}
            \item Dispatcher thread: 
                \begin{enumerate}
                    \item Monitor train status: monitor the status of each train
                        including loading, ready, granted, crossing, and gone.
                    \item Decision making: make decisions to let a train cross
                        depending on factors such as priority, and direction.
                    \item
                        Rule enforcement: enforce simulation rules, such as
                        only allowing loaded trains the ability to cross, 
                        prioritizing trains.
                    \item
                        Communicate with train threads through mutex's, and convars.
                    \item
                        Handle special cases such as, two trains are loaded and
                        have high priority. 

                \end{enumerate}
            \item Train threads:
                \begin{enumerate}
                    \item Load time use usleep() to simulate the time it takes
                        to load. 
                    \item Wait for permission to cross from the dispatcher 
                        using mutex's and convars.
                    \item
                        Update status and notify dispatcher.
                \end{enumerate}
        \end{enumerate}
    \item
        \begin{enumerate}
            \item The threads do not work independently. There is a controller thread
                call the dispatcher thread that coordinates with each train 
                using mutex's and convars to grant or restrict access to the 
                main crossing track.
        \end{enumerate}
    \item
        \begin{enumerate}
            \item Station Mutex: Guards access to the station data structure.
            \item Train Mutex: Guards access to individual train data.
            \item Queue Mutex: Ensures that modification to the queue, such
                as adding or removing trains, are synchronized. 
            \item Crossing Mutex: Guards access to the main track during train crossings. 
            \item Direction Mutex: Guards access to information about the last
                direction of the train that crossed the main track.
        \end{enumerate}
    \item
        \begin{enumerate}
            \item No the main thread will not be idle.
            \item The main thread will communicate with the dispatcher thread
                with mutex's and convars. And update the state of 
                different attributes. For example it might change the stations
                status from loading to ready.
        \end{enumerate}
    \item 
        \begin{enumerate}
            \item Priority queue.
        \end{enumerate}
    \item 
        \begin{enumerate}
            \item Identify shared data structures.
            \item Assign mutexes.
            \item Use mutexes to guard critical sections.
            \item Avoid deadlocks.
        \end{enumerate}
    \item 
        \begin{enumerate}
            \item Convar for Train Loading:
                Represents the \textbf{condition} that a train has finished loading and
                ready to depart.\textbf{Mutex} associated with the station mutex,
                ensuring access to the station's loading status. This ensures that 
                the loading condition is checked and can be modified in a 
                synchronized manner. \textbf{Operation} once it
                has been unblocked, the train thread should re-check the 
                loading condition (loading status) and signal that it has been
                granted permission to cross.

            \item Convar for Train Crossing Permission:
                \textbf{Condition} Represents the condition that a train has been
                granted permission to cross the main track.
                \textbf{Mutex} Associated with the crossing mutex, 
                guarding access to the main track during train crossings.
                This ensures that the crossing condition is check and modified 
                when a train is supposed to cross or not.
                \textbf{Operation} once it has been unblocked, the train thread
                should check the crossing condition and proceed to cross the 
                main track.
            \item Convar for Queue Modification:
                \textbf{Condition} Represents the condition that the station queue has been modified
                (a train has been added or removed).
                \textbf{Mutex} Associated with the queue mutex, controlling access
                to the stations queue. This ensures that queue modifications
                are synchronized.
                \textbf{Operation} Once pthread\_cond\_wait() has been unblocked,
                the train thread should re-check the queue modification condition.
                If there has been a modification like a train has been added or 
                removed from the queue then the position of the current train 
                on the queue will be calculated.
            \item Convar for Direction Decisions:
                \textbf{Condition} Represents the condition that the 
                direction of the next train has been made.
                \textbf{Mutex} Associated with the direction mutex, 
                guarding information about the last direction of 
                the train that crossed the main track. This ensures
                that direction decisions are synchronized.
                \textbf{Operation} Once it has been unblocked the train thread
                should re-check the direction condition, and make a decision 
                based on what direction the next train should go.
        \end{enumerate}
    \item
        \begin{enumerate}
            \item If a train is loaded, acquire the station mutex to safely modify station data.
            \item Add the loaded train to the station's queue for departure.
            \item Release the station mutex to allow other threads to access shared station resources.
            \item Train threads wait for permission from the dispatcher to cross the main track.
            \item Dispatcher monitors statuses, applies rules, and grants permission based on priorities.
            \item Simulate train crossings, update statuses, and notify the dispatcher when done.
            \item Print simulation output: train arrivals, crossings, and departures with proper timing.
        \end{enumerate}
\end{enumerate}
\end{document}

